\documentclass{article}

\usepackage{colortbl}
\usepackage{listings}
\usepackage{todonotes}
\usepackage{tikz}

% Let's check if bing AI memory layout generator works 
\newcommand{\memorymap}[3]{
  #1 & #2 & #3 \\
  \hline
}

\title{Yasboot memory layout}
\author{Mateusz Stadnik}
\date{\today}

\begin{document}
\maketitle

\section{Introduction}
This document describes Yasboot memory layout. 
Be careful with using it, each MCU or board may have slightly different exact mapping. 
For details about specific MCU, please check dedicated document. 

\section{Memory Map}
Yasboot uses MBR layout scheme where it is possible. 
Yasboot second stage is placed just after MBR block. 
Size of second stage is highly configurable via: 
\begin{lstlisting}[language=bash]
$ make menuconfig
\end{lstlisting}
\todo{Add pictures about Yaboot configure options}

\noindent Generic memory layout: 

\begin{tabular}{|c|c|c|}
  \hline
  \rowcolor{gray!50} 
  \memorymap{Section Name}{Section Address}{Section Size} 
  \rowcolor{yellow!20}
  \memorymap{MBR}{0x00000000}{0x200}
  \rowcolor{yellow!20}
  \memorymap{Yaspem 2-stage}{0x00000200}{0x800...0x8000}

  \rowcolor{green!10}
  \memorymap{Optional BIOS}{0x00000200 + 2stage size}{0x4000}
  \rowcolor{blue!5} 
  \memorymap{1-st partition}{user defined}{user defined}
  \rowcolor{blue!5} 
  \memorymap{...}{...}{...}
  \rowcolor{blue!5} 
  \memorymap{n-th partition*}{user defined}{user defined}
\end{tabular}

* where \textbf{n} maximum is 4 according to MBR scheme.

\end{document}
